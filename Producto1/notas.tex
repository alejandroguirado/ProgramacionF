Cat
Cat , ver el contenido de una archivo.
$ cat prueba.txt
Ls
Ls (de listar), permite listar el contenido de un directorio o fichero.
$ ls /home/directorio
Ls -la
El comando ls tiene varias opciones que permiten organizar la salida.
$ ls -la /home/directorio
Cd
Cd (de change directory o cambiar directorio) para acceder a una ruta distinta de la que te encuentras.
$ cd /home/ejercicios
Touch
Touch crea un archivo vacío.
$ touch /home/chess.txt
Mkdir
Mkdir (de make directory o rear directorio), crea un directorio nuevo tomando en cuenta la ubicación actual.
$ mkdir /home/ejercicios
Mkdir -p
Mkdir crear un árbol de directorios completo que no existe.
$ mkdir -p /home/ejercicios/prueba/uno/dos/tres
Cp
Cp (de copy o copiar), copia un archivo o directorio origen a un archivo o directorio destino.
$ cp /home/prueba.txt /home/respaldo/prueba.txt
Cp-r
Copiar al directorio y a todos los subdirectorios.
$ cp -r /home/ejercicios /home/respaldos/
Mv
Mv (de move o mover), mueve un archivo a una ruta específica, y a diferencia de cp, lo elimina del origen finalizada la operación.
$ mv /home/prueba.txt /home/respaldos/prueba2.txt
Rm (de remove o remover), es el comando necesario para borrar un archivo o directorio.
$ rm /home/prueba.txt
Pwd (de print working directory o imprimir directorio de trabajo). Imprime la ruta especificada.
$ pwd
Clear
Clear (de limpiar), es un sgrepencillo comando que limpiara nuestra terminal por completo dejándola como recién abierta.
$ clear
Less
Muestra menos del contenido en pantalla.
$ less alibaba.txt
Head
Escribe las primeras 10 lineas del archivo.
$ head chess.txt
Tail
Escribe las ultimas diez lineas del archivo.
$ tail chess.txt
Wc (word count)
Para contar palabras de un texto.
$ wc chess.txt
Grep
Buscar palabras en algun archivo de texto
$ grep king chess.txt
Whoami
Muestra el usuario que ha iniciado sesion en la terminal
$ whoami
Cat
Cat , ver el contenido de una archivo.
$ cat prueba.txt
Ls
Ls (de listar), permite listar el contenido de un directorio o fichero.
$ ls /home/directorio
Ls -la
El comando ls tiene varias opciones que permiten organizar la salida.
$ ls -la /home/directorio
Cd
Cd (de change directory o cambiar directorio) para acceder a una ruta distinta de la que te encuentras.
$ cd /home/ejercicios
Touch
Touch crea un archivo vacío.
$ touch /home/chess.txt
Mkdir
Mkdir (de make directory o rear directorio), crea un directorio nuevo tomando en cuenta la ubicación actual.
$ mkdir /home/ejercicios
Mkdir -p
Mkdir crear un árbol de directorios completo que no existe.
$ mkdir -p /home/ejercicios/prueba/uno/dos/tres
Cp
Cp (de copy o copiar), copia un archivo o directorio origen a un archivo o directorio destino.
$ cp /home/prueba.txt /home/respaldo/prueba.txt
Cp-r
Copiar al directorio y a todos los subdirectorios.
$ cp -r /home/ejercicios /home/respaldos/
Mv
Mv (de move o mover), mueve un archivo a una ruta específica, y a diferencia de cp, lo elimina del origen finalizada la operación.
$ mv /home/prueba.txt /home/respaldos/prueba2.txt
Rm (de remove o remover), es el comando necesario para borrar un archivo o directorio.
$ rm /home/prueba.txt
Pwd (de print working directory o imprimir directorio de trabajo). Imprime la ruta especificada.
$ pwd
Clear
Clear (de limpiar), es un sgrepencillo comando que limpiara nuestra terminal por completo dejándola como recién abierta.
$ clear
Less
Muestra menos del contenido en pantalla.
$ less alibaba.txt
Head
Escribe las primeras 10 lineas del archivo.
$ head chess.txt
Tail
Escribe las ultimas diez lineas del archivo.
$ tail chess.txt
Wc (word count)
Echo
Se encarga de repetir o desplegar en la salida estándr cualquier argumento que se le indíque(inclusive comodínes), para posteriormente saltar una línea.
$ solo para ya!
